\chapter*{Philosophie du document}


\prof{\textbf{ceci est la version professeur du document.} L'icône du professeur est suivie par des informations complémentaires qui n'apparaissent pas dans la version élève.}  

Vous avez entre les mains le premier exemplaire d'une série de quatre fascicules qui accompagneront les élèves des classes de 6\up{e}, 5\up{e}, 4\up{e} et 3\up{e} dans leur découverte et maîtrise de l'outil informatique.

\vspace{18pt}

Ce document se présente sous la forme d'un fascicule qui rassemble des fiches MITIC\footnote{MITIC : Médias, Images et Technologies de l'Information et de la Communication.} permettant aux élèves d'apprendre à utiliser les logiciels et espaces numériques mis à leur disposition. Pour l'année de 6\up{e}, sont traités les logiciels \emph{Microsoft Word} (traitement de texte), \emph{Microsoft Excel} (tableur grapheur), \emph{Gimp} (retouche d'image), \emph{Scratch} (programmation) ainsi que l'outil \emph{Microsoft Teams} présent sur l'espace numérique de travail (ENT) de notre école. %Le choix s'est porté sur des logiciels libres et gratuits, multiplateformes, aisément disponibles sur l'internet.

\vspace{18pt}


Chaque fiche est conçue pour être exploitée à trois occasions et dans trois matières différentes, à chaque fois lors d'une séance de 45 minutes. La fiche sur le tableur, par exemple, est découverte en mathématiques (\emph{Séance 1}), exploitée à nouveau en physique-chimie (\emph{Séance 2}) puis en histoire-géographie (\emph{Séance 3}) selon un calendrier proposé en début de fiche. Nous avons à chaque fois essayé de faire coïncider les notions abordées dans la fiche avec le programme de la matière concernée. Remarque : les séances 2 et 3 peuvent être inversées si nécessaire, puisqu'elles reprennent les notions découvertes dans la première séance de la fiche.

\vspace{18pt}

Au début de l'année, chaque titulaire de 6\up{e} doit emmener les élèves dont il a la charge en salle informatique et leur faire découvrir la plateforme \emph{Teams} (fiche page \pageref{teams1}). Au cours de l'année, les professeurs de chaque matière concernée par une fiche sont responsables de sa réalisation avec les élèves.

\vspace{18pt}

Professeurs, c'est à vous que revient la tâche délicate d'inclure le contenu de ces fiches dans votre progression. À vous de le faire vivre : arriver en salle informatique et demander aux élèves de remettre en forme un texte de Jonathan Swift ne présente que peu d'intérêt pédagogique. Donnez du sens à ces fiches et profitez-en pour diversifier votre enseignement. N'hésitez pas à exploiter dans vos cours les techniques présentées dans ce fascicule afin que les élèves utilisent plusieurs fois leurs nouvelles compétences et, par là-même, les pérennisent.

\vspace{18pt}

À la fin de ce fascicule sont proposées des idées d'activités supplémentaires sous forme de projet : ainsi les élèves exploiteront-ils les connaissances acquises au cours de l'année. Ces activités pluridisciplinaires permettent aussi de faire comprendre aux élèves que les cours qu'ils suivent ne sont pas des entités cloisonnées, mais, qu'à l'inverse, les compétences qu'ils y développent sont transposables d'une matière à une autre. Dès lors, ils comprendront que ce qu'ils étudient en classe fait partie d’un socle de connaissances transdisciplinaires utiles et nécessaires à leur avenir.

\vspace{18pt}

%Ces fiches MITIC sont appelées à évoluer. N'hésitez pas à nous transmettre vos suggestions et nous signaler toute erreur relevée par courriel à l'adresse \texttt{flo-mitic@florimont.ch}.

%\vspace{18pt}

Merci d'avance à tous pour votre implication.

\vspace{18pt}

L'équipe de rédaction.

\vspace{2cm}

%\emph{Remarque : il existe une version professeur de ce document, contenant des informations complémentaires, disponible sur l'ENT de l'école.}

  
