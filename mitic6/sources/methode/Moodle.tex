\chapter{La plateforme Flore}\label{plateformeFlore}  

\begin{itemize}
\item Logiciel : aucun
\item Prérequis : aucun
\item Matières concernées : fiche réalisée par le titulaire
\item Objectifs : savoir se connecter à la plateforme \emph{Flore}, accéder aux plateformes \emph{Pronote} et \emph{Moodle}, remettre un devoir sur Moodle.
%\item Compétences : 
%        \begin{itemize}
%        \item insérer une formule
%        \end{itemize}
\item Cette fiche est à réaliser :
        \begin{itemize}
        \item au plus vite, dès le début de l'année. 
        \end{itemize}
\end{itemize}

\vspace{12pt}

La plateforme numérique de l'institut Florimont s'appelle \emph{Flore}. En vous connectant à \emph{Flore}, vous pouvez accéder à \emph{Pronote} (contient l'emploi du temps, le cahier de texte de la classe, les notes et les informations en provenance de l'école) et à \emph{Moodle} (qui est un espace d'échange entre les élèves et leurs professeurs, qui peut contenir des supports de cours, des activités, et où on peut remettre des devoirs, etc.).

\section{Se connecter à la plateforme \emph{Flore}}\label{FloreConnexion}\index{Connexion à Flore}\index{Flore!Se connecter}

Pour se connecter à la plateforme \emph{Flore}, il faut se rendre sur le site web de l'école à l'adresse \url{http://www.florimont.ch}. Il faut ensuite cliquer sur le bouton \texttt{Flore} en haut à droite de la page :

\uneimageici{./images/methode/FloreAcces1}{\textwidth}

Dans la page qui s'ouvre, il faut entrer l'identifiant (le \emph{login} ou \emph{username}), ainsi que le mot de passe (le \emph{password}) qui vous ont été fournis par votre titulaire en début d'année.


\uneimageici{./images/methode/FloreAcces2}{.5\textwidth}

La page suivante permet de se connecter aux plateformes \emph{Pronote} ou \emph{Moodle}.



\section{Se connecter à la plateforme \emph{Pronote}}\index{Connexion à Pronote}\index{Pronote!Se connecter}

Une fois connecté à \emph{Flore} (voir paragraphe \vref{FloreConnexion}), en cliquant sur l'icône \texttt{Pronote}, vous accédez à votre espace personnel \emph{Pronote}.

\uneimageici{./images/methode/PronoteAcces1}{.5\textwidth}

Dans l'espace personnel \emph{Pronote}, vous retrouvez votre emploi du temps de la journée, le travail à faire (devoirs), les dernières notes obtenues, et les informations générales de l'école.

\uneimageici{./images/methode/PronotePresentation}{\textwidth}









\section{Se connecter à la plateforme \emph{Moodle}}\index{Connexion à Moodle}\index{Moodle!Se connecter}

Une fois connecté à \emph{Flore} (voir paragraphe \vref{FloreConnexion}), en cliquant sur l'icône \texttt{Moodle}, vous accédez à votre espace personnel \emph{Moodle}.

\uneimageici{./images/methode/MoodleAcces1}{.5\textwidth}

L'espace personnel \emph{Moodle} contient les pages de vos différents cours ainsi qu'un espace élève :

\uneimageici{./images/methode/MoodleAcces2}{\textwidth}






\section{Récupérer un document sur la plateforme \emph{Moodle}}\label{MoodlePrendreDoc}\index{Récupérer un document sur Moodle}\index{Moodle!Récupérer un devoir}

Après avoir accédé à la page correspondant au cours (par exemple ci-dessous la page \texttt{6.F3\_MATHEMATIQUES}), repérer le fichier à récupérer préparé par l'enseignant puis cliquer dessus :

\uneimageici{./images/methode/MoodleRecupererFichier1}{.6\textwidth}

La boîte de dialogue \texttt{Ouverture de ...} s'ouvre alors. Il faut choisir \texttt{Enregistrer le fichier}.

\uneimageici{./images/methode/MoodleRecupererFichier2}{.6\textwidth}

Votre fichier est enregistré automatiquement dans le dossier \texttt{Téléchargement} de l'ordinateur. Pour le récupérer, il faut ouvrir le \emph{Finder}.\index{Ouvrir!Finder}\index{Finder!Ouvrir}

Lancer le logiciel en utilisant la <<\,loupe\,>> :

\uneimageici{./images/generales/loupe}{.7\textwidth}

... puis en indiquant \emph{Finder} :

\uneimageici{./images/generales/loupeRechercheFinder}{.7\textwidth}

La fenêtre principale du \emph{Finder} s'ouvre. Sur la gauche de la fenêtre, parmi les \emph{Favoris}, se trouve le dossier \emph{Téléchargements}. Cliquer sur le fichier puis, en maintenant le clic, tirer et déposer le fichier sur le dossier \emph{Bureau}. Le fichier est alors déplacé vers le \emph{Bureau} de l'ordinateur.  

\uneimageici{./images/methode/MoodleRecupererFichier3}{\textwidth}

 





\section{Remettre un devoir sur la plateforme \emph{Moodle}}\label{MoodleRendreDevoir}\index{Remise d'un devoir sur Moodle}\index{Moodle!Remettre un devoir}

Après avoir accédé à la page correspondant au cours (par exemple ci-dessous la page \texttt{6.F3\_MATHEMATIQUES}), repérer le dossier de remise de devoir préparé par l'enseignant, signalé par l'icône \includegraphics[width=.04\textwidth]{./images/methode/MoodleDevoirIcone1} : cliquer dessus.

\uneimageici{./images/methode/MoodleDevoir1}{.8\textwidth}

Cliquer ensuite sur le bouton \texttt{Remettre un devoir} :

\uneimageici{./images/methode/MoodleDevoir2}{.8\textwidth}

Deux solutions sont possibles pour remettre un devoir :
\begin{itemize}
\item méthode la plus simple : faire glisser le fichier contenant votre devoir dans la zone prévue à cet effet (repérée par la flèche \includegraphics[width=.04\textwidth]{./images/methode/MoodleDevoirIcone2}) ;
\item autre méthode : cliquer sur le bouton \texttt{Ajouter...} (icône \includegraphics[width=.04\textwidth]{./images/methode/MoodleDevoirIcone3}).
\end{itemize}

\uneimageici{./images/methode/MoodleDevoir3b}{.8\textwidth}

Cliquer alors sur \texttt{Importer un fichier de mon Finder} :

\uneimageici{./images/methode/MoodleDevoir4}{.8\textwidth}

Puis dans la page suivante, sur \texttt{Parcourir} :

\uneimageici{./images/methode/MoodleDevoir5}{.5\textwidth}


Terminer le dépôt en cliquant si nécessaire sur le bouton \texttt{Déposer ce fichier}.












 

