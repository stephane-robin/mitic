\chapter{Activités supplémentaires}

\prof{vous trouverez ci-dessous quelques idées de projets que vous pouvez proposer aux élèves. Le but de ces projets est de leur permettre d'utiliser les différentes compétences acquises au cours de l'année scolaire pour composer un document où peuvent se mêler différentes matières et différents média.}

 

\section{Préparer un compte rendu d'expérience en science}

Préparer un compte rendu d'expérience, au format PDF, qui comprenne :

\begin{itemize}
\item un texte écrit et mis en forme avec \emph{Writer} ;
\item des images retravaillées avec \emph{Gimp} qui montre le schéma du montage utilisé ou les résultats obtenus ;
\item des graphiques réalisés avec \emph{Calc} pour montrer les données collectées lors de l'expérience.
\end{itemize}








\section{Élaborer un document en histoire-géographie}

\prof{avant de réaliser cette activité, il faut mettre à disposition sur la page \emph{Moodle} de votre cours :\begin{itemize}\item le texte non mis en forme ;\item l'image à retravailler.\end{itemize}. Il faut également préparer un dossier de remise de devoir pour que les élèves puissent rendre le document PDF final.}


Le but de cette activité est de composer un document au format PDF comprenant :
\begin{itemize}
\item un texte mis en forme avec \emph{Writer} ;
\item un graphique réalisé avec \emph{Calc} (voir données à traiter et modèle du graphique ci-dessous) ;
\end{itemize}
\deuximagesici{./images/activites/ActiviteParisTableauPopulation}{.7\textwidth}%
              {./images/activites/ActiviteParisGraphiquePopulation}{\textwidth}



\begin{itemize}
\item une image de la ville de Paris à recadrer avec \emph{Gimp}.
\uneimageici{./images/activites/ActiviteParisImageVille}{.8\textwidth}

\end{itemize}

\vspace{12pt}

Les documents nécessaires pour réaliser cette activité sont disponibles sur la plateforme \emph{Moodle}.

\vspace{12pt}
\textsl{Sources :\begin{itemize}\item \url{http://paris-atlas-historique.fr/}, accédé le 11 juillet 2016 ;\item \url{http://laurentfaes-geo.blogspot.ch/}, accédé le 11 juillet 2016.\end{itemize}}


\section{Construire un programme en \emph{Scratch}}

Un élève a écrit en scratch le programme suivant:

\uneimageici{./images/activites/ScratchAllerPlusLoinImage1}{.4\textwidth}

Le but de cette activité est de:
\begin{itemize}
\item deviner ce que fait le programme donné;
\item recopier le programme pour vérifier le résultat;
\item programmer un second script pour ce lutin qui réalise le tracé symétrique par rapport à l'origine du repère. Ce second script commencera par les instructions suivantes:
\end{itemize}
\uneimageici{./images/activites/ScratchAllerPlusLoinImage2}{.4\textwidth}

