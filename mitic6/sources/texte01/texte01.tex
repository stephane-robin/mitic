

\newpage

\section{Aide pour réaliser les activités}
\subsection{\label{correction_texte01}Aide pour la Séance 1}

\subsubsection{Mettre en forme la page}\index{Writer!Marges}\index{Marges (Writer)}

Pour commencer on définit la taille des quatre marges du document (gauche, droite, haut et bas). On choisit ici 4\,cm à gauche et à droite, et 3\,cm en haut et en bas.

Dans le menu \texttt{Mise en page}, choisir \texttt{Marges}, puis \texttt{Marges personnalisées...}:    

\uneimageici{./images/texte/forme1.png}{.48\textwidth}

Dans la boîte de dialogue qui s'ouvre, se rendre dans l'onglet \texttt{Marges} puis régler les marges :  

\uneimageici{./images/texte/forme2.png}{.7\textwidth}

Vous pouvez éventuellement vous rendre dans l'onglet \texttt{Mise en page} puis en profiter pour régler les \texttt{En-têtes} et \texttt{Pieds de page}.











\subsubsection{Choisir une police de caractères}\index{Writer!Changer la police de caractères}\index{Police de caractère (Writer)}

Par défaut, la police de caractères du document est \texttt{Calibri} avec une taille de 12. On souhaite passer à \texttt{Times New Roman}. La première étape est de sélectionner tout le texte. Pour cela :

\begin{itemize}
\item soit, on passe par le menu \texttt{Édition} et on choisit \texttt{Sélectionner Tout} ;
\item soit, on utilise le raccourci clavier \texttt{cmd + A}.\index{Raccourci Clavier! Cmd + A, sélectionner tout (Writer)}    
\end{itemize}

\deuximagesici{./images/texte/police1.png}{.7\textwidth}
              {./images/generales/clavierCmdA}{.7\textwidth}

Dans la liste déroulante des polices de caractères (1 dans l'image ci-dessous), choisir la police \texttt{Times New Roman} : le texte sélectionné change alors de police de caractères.  


\uneimageici{./images/texte/police2.png}{.7\textwidth}


\subsubsection{Modifier la taille des caractères}\index{Writer!Taille des caractères}\index{Tailles des caractères (Writer)} 

Avant de modifier la taille des caractères, il faut sélectionner les parties du texte à modifier. On choisit une taille de 12 points pour le texte (on peut utiliser \texttt{cmd + A} pour tout sélectionner ou passer par le menu \texttt{Édition}) et 16 pour le titre (à sélectionner à la souris avant le changement de taille).

\deuximagesici{./images/texte/taille1.png}{.7\textwidth}%
              {./images/texte/taille2.png}{.7\textwidth}







\subsubsection{Centrer du texte}\index{Writer!Centrer le texte}\index{Centrer le texte (Writer)}

Pour centrer les trois premières lignes, il faut tout d'abord les sélectionner à l'aide de la souris. Cliquer ensuite sur le bouton \emph{Centrer le texte} : 

\uneimageici{./images/texte/centrer1.png}{.6\textwidth}

%%%%%%%%%%%%%%%%%%%%%%%%%%%%%%%%%%%%%%%%%%%%%%%%%%%%%%%%%%
%\subsection{Créer une liste à puces}\index{Writer!Liste à puces}\index{Liste à puces (Writer)}   
%
%Les différents personnages de l'acte 1 scène 1 sont présentés sous la forme d'une liste où chaque ligne démarre par un $\bullet$ qui s'appelle une \emph{puce}. L'ensemble est nommé \emph{liste à puces}.   
%
%Pour créer une liste à puce, sélectionner les lignes concernées puis cliquer sur le bouton \emph{Activer les puces}.
%
%\uneimageici{./images/texte/WriterListePuce}{.7\textwidth}
%%%%%%%%%%%%%%%%%%%%%%%%%%%%%%%%%%%%%%%%%%%%%%%%%%%%%%%%%%%




\subsubsection{Souligner du texte}

On souligne le titre.\index{Writer!Souligner}\index{Souligner (Writer)} Pour cela, on le sélectionne, puis on applique une des deux méthodes suivantes :

\begin{itemize}
\item utilisation du bouton \emph{souligner} :  
\uneimageici{./images/texte/souligner1.png}{.7\textwidth}
\item utilisation du raccourci clavier \texttt{cmd + U} :\index{Raccourci Clavier! Cmd + U, souligner (Writer)} 
\uneimageici{./images/generales/clavierCmdU}{.3\textwidth}
\end{itemize}




\subsubsection{Mettre un texte en gras ou en italique}

Pour mettre du texte en gras ou en italique\index{Writer!Gras}\index{Gras (Writer)}\index{Writer!Italique}\index{Italique (Writer)}, il faut tout d'abord le sélectionner, puis appliquer une des deux méthodes suivantes :

% attention, ci-dessous j'ai fait deux listes à puces différentes pour pouvoir conserver 
% une pleine largeur pour les images...
\begin{itemize}
\item utilisation du bouton \emph{gras} ou \emph{italique} :
\end{itemize}
\deuximagesici{./images/texte/gras1.png}{\textwidth}%
              {./images/texte/italique1.png}{\textwidth}
\begin{itemize}
\item utilisation du raccourci clavier \texttt{cmd + B} (gras) ou \texttt{cmd + I} (italique)\index{Raccourci Clavier! Cmd + I, italique (Writer)}\index{Raccourci Clavier! Cmd + B, gras (Writer)} :
\end{itemize}
\deuximagesici{./images/generales/clavierCmdB}{.7\textwidth}%
              {./images/generales/clavierCmdI}{.7\textwidth}


    









\subsubsection{Modifier la couleur des caractères} 


On change la couleur du titre\index{Writer!Changer la couleur des caractères}\index{Couleur des caractères (Writer)}. Pour cela, on le sélectionne, puis on choisit la couleur désirée (par exemple \emph{Rouge foncé}) :

\uneimageici{./images/texte/couleur1.png}{.4\textwidth}










\subsubsection{Mettre en forme des paragraphes}\index{Writer!Interligne}\index{Writer!Espacement entre paragraphes}\index{Interligne (Writer)}\index{Espacement entre paragraphes (Writer)} 



Il faut tout d'abord sélectionner tous les paragraphes sur lesquels on veut appliquer un changement.

Comme l'espace entre les lignes est une propriété du paragraphe, on peut le modifier en passant par le menu \texttt{Mise en forme} puis \texttt{Paragraphe...}

On va modifier deux propriétés : 
\begin{itemize}
\item l'espace entre deux lignes au sein d'un même paragraphe (\emph{interligne}) ;
\item l'espace entre deux paragraphes (\emph{espacement sous le paragraphe}).
\end{itemize}


\uneimageici{./images/texte/paragraphe1.png}{.7\textwidth}

Dans la boîte de dialogue qui s'ouvre, se rendre dans l'onglet \texttt{Retrait et espacement}, puis régler l'interligne et l'espacement sous le paragraphe :  

\uneimageici{./images/texte/paragraphe2.png}{.6\textwidth}







\subsubsection{Justifier un paragraphe}\index{Writer!Justifier}\index{Justifier (Writer)}\index{Aligner le texte à droite et à gauche (justifier) (Writer)}

Lorsque le texte est aligné à la fois du côté gauche et du côté droit, on dit qu'il est \emph{justifié}. Pour justifier les parties du texte qui doivent l'être:

\begin{itemize}
\item sélectionner les paragraphes à justifier à l'aide de la souris
\item cliquer sur le bouton \emph{justifié} ou utiliser le raccourci clavier \texttt{cmd + J}. 
\end{itemize}

\uneimageici{./images/texte/justifier.png}{.7\textwidth}





\subsubsection{Aligner à droite}  

Pour terminer, on aligne à droite les points de suspension finaux. Pour cela, il faut les sélectionner à l'aide de la souris puis cliquer sur le bouton \texttt{Aligner à droite} :  

\uneimageici{./images/texte/alignerDroite.png}{.5\textwidth}



\subsubsection{Exporter au format PDF}\index{Writer!Exporter au format PDF}\index{PDF (exporter au format) (Writer)}

Une fois le travail achevé et sauvegardé, il faut exporter le fichier au format PDF. Pour cela, il faut 
passer par le menu \texttt{Fichier} et choisir \texttt{Enregistrer sous...} à la place de \texttt{Enregistrer}. Il faut ensuite choisir \texttt{Formats d'exportation - PDF} \textit{(numéro 2 dans l'image ci-dessous)}, puis l'emplacement d'enregistrement, comme par exemple \texttt{le Bureau}. Terminez l'exportation en cliquant sur \texttt{Enregister} \textit{(numéro 3 dans l'image ci-dessous)}.

\deuximagesPGici{./images/texte/export1.png}{\textwidth}%
                {./images/texte/export2.png}{\textwidth}



Le fichier PDF est alors enregistré au même endroit que le fichier sur lequel on travaille.


\cadre{Le \textbf{format PDF} est un format parfaitement adapté aux échanges de documents : on ne peut le modifier sans laisser la trace de ce changement, et il est lisible sur tous les périphériques (ordinateurs, tablettes, smartphones) en conservant son aspect initial. Il peut contenir du texte, des images, des liens vers l'internet et même des vidéos ou du son. À chaque fois qu'il faut rendre ou envoyer un document qui n'est pas destiné à être modifié, il faut privilégier le format de fichier PDF.}  








\subsubsection{Remettre le travail achevé sur Teams}

Une fois votre travail terminé et exporté au format PDF, il faut le remettre au professeur. Pour cela, se connecter à la page Teams du cours. Chercher le dossier de remise de devoir, puis remettre le travail. Si nécessaire, se reporter à la fiche méthode \emph{Remettre son devoir}, page \pageref{TeamsRemettreDevoir}.  








%%%%%%%%%%%%%%%%%%%%%%%%%%%%%%%%%%%%%%%%%
%\subsubsection{Remplacer les : par des .}\index{Writer!Rechercher et remplacer}\index{Rechercher et remplacer (Writer)}
%
%\uneimageici{./images/texte/WriterEditionMenu}{.35\textwidth}
%
%\uneimageici{./images/texte/WriterBoiteRechercherRemplacer}{.5\textwidth}
%%%%%%%%%%%%%%%%%%%%%%%%%%%%%%%%%%%%%%%%%

\poubelle{

%
%
%  S  É  A  N  C  E     II
%
%


\subsection{Aide pour la Séance 2}\label{correction_texte02}


\subsubsection{Passer le texte en majuscule}\index{Writer!Majuscule}\index{Majuscule (Writer)}

Certaines parties du texte doivent être écrites en majuscules. Pour cela, sélectionner-les\footnote{On peut sélectionner différents endroits du texte en même temps en maintenant la touche \texttt{Cmd} enfoncée pendant qu'on sélectionne les zones à la souris.} puis ouvrir le menu \texttt{Format} et choisir \texttt{Caractère...} Une boîte de dialogue s'ouvre : dans l'onglet \texttt{Effet de caractère}, il faut choisir \texttt{Majuscules}, puis cliquer sur \texttt{OK}.  

\uneimageici{./images/texte/WriterTexteMajuscule}{.9\textwidth}









\subsubsection{Encadrer un paragraphe}\index{Writer!Encadrer}\index{Encadrer (Writer)} 

Pour encadrer le titre de l'ouvrage et ajouter un espacement entre le texte et la bordure qui l'entoure, on utilise l'outil \emph{bordures de paragraphe}. Pour cela, sélectionner le titre, puis ouvrir le menu \texttt{Format} et choisir \texttt{Paragraphe...} Une boîte de dialogue s'ouvre : dans l'onglet \texttt{Bordures}, modifier les propriétés des bordures comme indiqué ci-dessous, puis cliquer sur \texttt{OK}.

\uneimageici{./images/texte/WriterParagrapheEncadre}{.9\textwidth}



\subsubsection{Pour aller plus loin...}

Une fois votre document rendu au format PDF sur la page Teams de votre cours, amusez-vous à découvrir toutes les propriétés des paragraphes. Vous pouvez par exemple ajouter :
\begin{itemize}
\item un retrait pour le premier mot du paragraphe ;
\item un retrait pour le paragraphe en entier ; 
\item un arrière plan (une couleur de fond) ;
\item une ombre derrière le paragraphe ; 
\item divers types de bordures. 
\end{itemize}


%
%
%  S  É  A  N  C  E     III
%
%


\subsection{Aide pour la Séance 3}\label{correction_texte03}



\subsubsection{Insérer une image}\index{Writer!Insérer une image}\index{Insérer une image (Writer)}

La première étape est de récupérer sur la page Teams du cours l'image à insérer dans le compte rendu. Si nécessaire, se reporter à la fiche méthode \emph{Récupérer un document sur Teams}, paragraphe \vref{MoodlePrendreDoc}.

\vspace{12pt}

Pour insérer une image dans un document texte :

\begin{enumerate}
\item Cliquer sur la souris pour positionner le curseur à l'endroit où l'image doit être insérée et sauter quelques lignes pour laisser de la place à l'image :

\uneimageici{./images/texte/WriterInsertionImage2}{.6\textwidth}

\item Se rendre dans le menu \texttt{Insertion} et choisir \texttt{Image...}

\uneimageici{./images/texte/WriterInsertionImage1}{.6\textwidth}

\item Dans la boîte de dialogue qui s'ouvre, rechercher le fichier qui contient l'image et terminer l'insertion en cliquant sur \texttt{Ouvrir}.

\item L'image peut alors être redimensionnée :
        \begin{itemize}
        \item soit, en utilisant les poignées qui apparaissent lorsqu'on clique sur l'image ;
        \uneimageici{./images/texte/WriterInsertionImage3}{.5\textwidth}
        \item soit, en double-cliquant sur l'image pour faire apparaître la boîte de dialogue suivante. Il faut dans un premier temps cocher la case \texttt{Conserver le ratio}, puis on peut régler les dimensions souhaitées pour l'image.
        \uneimageici{./images/texte/WriterInsertionImage4}{.7\textwidth}
        \end{itemize}
\end{enumerate}


}