\vspace*{1cm}

\section*{Calendrier des différentes activités (6\up{e})}\index{Calendrier des activités}  

\vfill

\begingroup % permet de bloquer le arraystretch à ce groupe seulement
\renewcommand{\arraystretch}{1.2}
\begin{center}
\begin{tabular}{|l|l|c|l|l|}
\hline
\multirow{2}{*}{\textbf{Nom de la fiche}} & \multirow{2}{*}{\textbf{Matière}} & \multirow{2}{*}{\textbf{Page}} & \textbf{Date de} & \textbf{Nom du} \\
 &  &  & \textbf{réalisation} & \textbf{professeur} \\ \hline
\rowcolor[gray]{0.8}\multicolumn{5}{|l|}{Rentrée scolaire} \\ \hline 
\emph{Microsoft Teams} & (Titulaire) & \pageref{teams1} & & \phantom{xxxxxxxxxxxxxxxx}  \\ \hline
%
% avant les vacances d'octobre
%
\rowcolor[gray]{0.8}\multicolumn{5}{|l|}{Avant les vacances d'octobre} \\ \hline
\emph{Tableur : séance 1} & Mathématiques & \pageref{ficheTableur1} & & \\ \hline
\emph{Texte : séance 1} & Français & \pageref{ficheTexte1} & & \\ \hline
%
% avant les vacances de Noël
%
\rowcolor[gray]{0.8}\multicolumn{5}{|l|}{Avant les vacances de Noël} \\ \hline
\emph{Tableur : séance 2} & Physique-chimie & \pageref{ficheTableur3} & & \\ \hline
\emph{Texte : séance 2} & Anglais & \pageref{ficheTexte2} & & \\ \hline
\emph{Scratch : séance 1} & Mathématiques & \pageref{ficheScratch1} & & \\ \hline
%
% avant les vacances de février
%
\rowcolor[gray]{0.8}\multicolumn{5}{|l|}{Avant les vacances de février} \\ \hline
\emph{Présentation : séance 1} & Titulaire & \pageref{Presentation6eOutils} & & \\ \hline
%
% avant les vacances de printemps
%
\rowcolor[gray]{0.8}\multicolumn{5}{|l|}{Avant les vacances de printemps} \\ \hline
\emph{Texte : séance 3} & SVT & \pageref{ficheTexte3} & & \\ \hline
\emph{Scratch : séance 2} & Mathématiques & \pageref{ficheScratch2} & & \\ \hline
%
% avant les vacances d'été
%
\rowcolor[gray]{0.8}\multicolumn{5}{|l|}{Avant les vacances d'été} \\ \hline
\emph{Scratch : séance 3} & Mathématiques & \pageref{ficheScratch3} & & \\ \hline \hline
%
% avant la fin du semestre de cours
%
\rowcolor[gray]{0.8}\multicolumn{5}{|l|}{Avant la fin du semestre de cours (pour les cours au semestre)} \\ \hline
\emph{Tableur : séance 3} & Histoire-géographie & \pageref{ficheTableur2} & & \\ \hline
\emph{Image : séance 1} & Arts visuels & \pageref{ficheImage1} & & \\ \hline
\emph{Image : séance 2} & Français & \pageref{ficheImage2} & & \\ \hline
\emph{Image : séance 3} & Arts visuels & \pageref{ficheImage3} & & \\ \hline
\end{tabular}
\end{center}
\endgroup

\vfill
