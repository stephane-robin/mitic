\chapter{Programmation Scratch}\label{ficheScratch1}  

Les ordinateurs sont des machines qui exécutent des programmes. On peut écrire des programmes dans différents \emph{langages de programmation}, par exemple \emph{Python}, \emph{C++}, \emph{Java}... ou encore \emph{Scratch}.\\

\emph{Scratch} est un langage de programmation \textbf{visuelle} (on place des blocs d'instructions pour créer des programmes composés de codes) et \textbf{événementielle} (le programme réagit à des événements comme le clic de souris ou l'appui sur une touche). Il contient des \textbf{objets} : le lutin est un objet, l'arrière plan de la scène est un autre objet. On peut modifier les propriétés des objets, leur associer des codes, des costumes ou des sons. 

\section*{Synoptique}

{\footnotesize
\begin{itemize}
\item Logiciel\footnote{Le logiciel \emph{Scratch} est librement téléchargeable : \url{https://scratch.mit.edu/}} : \emph{Scratch 3.0}
\item Prérequis : aucun
\item Matière concernée : mathématiques
\item Objectifs : réaliser un programme simple en \emph{Scratch} (programme rendu sur \emph{Teams}).
\item Compétences : 
        \begin{itemize}
        \item choisir et paramétrer l'objet sprite ;
        \item choisir et paramétrer l'objet scène ;
        \item créer/insérer un nouvel objet ; 
        \item écrire un code comprenant mouvements, réponses à événement, boucles et son ;
        \item associer un code à un objet ;
        \item écrire un programme simple qui réponde à une problématique donnée.
        \end{itemize}
%\item Cette fiche est à réaliser :
       % \begin{itemize}
       % \item avant les vacances de Noël en mathématiques (séance 1) ;
        %\item avant les vacances de printemps en mathématiques (séance 2)  ;
        %\item avant les vacances d'été en mathématiques (séance 3). 
        %\end{itemize}
\prof{\item vous devez préparer un élément \underline{avant} la séance : \begin{itemize} \item créer un dossier de remise de devoir sur la page Teams de votre classe car les élèves rendent cette activité sous la forme d'un fichier PDF déposé sur Teams.\end{itemize}}
\end{itemize}
}







\newpage



%
%
%  S  É  A  N  C  E     I
%
%

\section{Séance 1 : un premier programme}

\subsection{Pour préparer l'activité...}

Voici une présentation vidéo de l'interface graphique de \texttt{Scratch}. Pensez à la regarder avant de commencer l'activité de la séance :

\begin{center}
\qrcode[hyperlink, height=0.9in]{https://web.microsoftstream.com/video/8ed24c7a-66c0-4ab6-9aa3-95ebdfc80ed6}
\end{center}




\subsection{Premiers pas avec Scratch...}\index{Ouvrir!Scratch}\index{Scratch!Ouvrir}

Lancer le logiciel en utilisant la <<\,loupe\,>> :

\uneimageici{./images/generales/loupe}{.7\textwidth}

... puis en indiquant \emph{Scratch} :

\uneimageici{./images/generales/loupeRechercheScratch}{.7\textwidth}


La fenêtre principale du logiciel s'ouvre. Elle se présente sous la forme de trois colonnes :

\begin{itemize}
\item la première colonne \circled{1} contient les différents blocs d'instructions que l'on utilise pour écrire les programmes ;
\item la seconde colonne \circled{2} est la zone dans laquelle sont construits les codes qui composent le programme ;
\item la troisième colonne \circled{3} contient deux zones :
        \begin{itemize}
        \item la zone où le programme s'exécute,
        \item la zone où on peut sélectionner les différents objets présents dans le programme.
        \end{itemize}
\end{itemize}

\uneimageici{./images/scratch/presentation_scratch.png}{\textwidth}

Si l'application \emph{Scratch} n'est pas installée sur votre ordinateur, la version en ligne vous sera proposée. En cliquant dessus, vous pourrez alors ouvrir une fenêtre similaire à celle présentée ci-dessus.

Vous pouvez dès à présent choisir la langue de l'interface en cliquant sur l'icone en haut à gauche  \includegraphics[width=1cm]{./images/scratch/changerLangue.png}. Nous choisirons par exemple \texttt{Français}

\uneimageici{./images/scratch/french.png}{.4\textwidth}





\subsubsection{Pour bien démarrer}


Dès que vous avez ouvert un nouveau document dans \emph{Scratch}, sauvegardez-le au format \texttt{Nom-seance1.sb3}  (dans le menu \texttt{Fichier}, choisir \texttt{Sauvegarder sur votre ordinateur}). Pendant que vous travaillez, pensez à sauvegarder régulièrement votre travail.   

%\uneimageici{./images/generales/clavierCmdS}{.4\textwidth}

\vfill
\phantom{rien}

\subsection{Sujet de l'activité...}

\vspace{10pt}

\boiteEnonce{Voici un premier programme : avant de l'écrire, essayez de deviner ce qu'il se passe lorsque le drapeau vert est pressé ! Le programme est volontairement écrit en anglais. A vous de retrouver les blocs correspondant en français. Recopiez ensuite le programme en cherchant à comprendre le rôle de chaque bloc de code. 
\uneimageici{./images/scratch/presentation_activite1.png}{.4\textwidth}
Une fois votre programme terminé, vous devrez l'enregistrer au format SB3 (le fichier doit être nommé à partir de votre nom : \texttt{Nom-seance1.sb3}) et le rendre sur Teams dans le dossier de remise de devoir, à l'endroit indiqué par votre professeur. Si nécessaire, se reporter à la fiche méthode \emph{Remettre son devoir}, page \pageref{TeamsRemettreDevoir}.}

\textbf{Pour obtenir de l'aide, rendez-vous à la page \pageref{correction_scratch1}}

\subsection{Pour aller plus loin...}

Rendez-vous dans la rubrique \texttt{Costumes} à côté des rubriques \texttt{Code} et \texttt{Sons}, et modifiez le costume de votre sprite à votre convenance, comme par exemple:

\uneimageici{./images/scratch/modifier_costume.png}{.2\textwidth}

\newpage

%
%
%  S  É  A  N  C  E     II
%
%









\section{Séance 2 : dessiner avec Scratch}\label{ficheScratch2}

\vspace{20pt}

\subsection{Pour préparer l'activité...}

Voici une vidéo pour apprendre à dessiner avec \texttt{Scratch}. Pensez à la regarder avant de commencer l'activité de la séance :

\begin{center}
\qrcode[hyperlink, height=0.9in]{https://web.microsoftstream.com/video/dcbece63-4f30-4ddc-95f1-83536a9e1356}
\end{center}



\vspace{20pt}

\subsection{Pour bien démarrer...}

Dès que vous avez ouvert un nouveau document dans \emph{Scratch}, sauvegardez-le au format \texttt{Nom-seance2.sb3}  (dans le menu \texttt{Fichier}, choisir \texttt{Sauvegarder sur votre ordinateur}). Pendant que vous travaillez, pensez à sauvegarder régulièrement votre travail.   

%\uneimageici{./images/generales/clavierCmdS}{.4\textwidth}

\vspace{20pt}

\subsection{Sujet de l'activité...}

\vspace{10pt}

\boiteEnonce{Lire le code suivant et essayer de deviner ce qu'il va se passer lorsque le programme est lancé. Construire ensuite le code en essayant de comprendre le rôle de chaque bloc de code. Le programme est volontairement écrit en anglais. A vous de retrouver les blocs correspondant en français.
}

\vfill
\phantom{rien}

\boiteEnonce{
\uneimageici{./images/scratch/presentation_activite2.png}{.4\textwidth}
Une fois votre programme terminé, vous devrez l'enregistrer au format SB3 (le fichier doit être nommé à partir de votre nom : \texttt{Nom-seance2.sb3}) et le rendre sur Teams dans le dossier de remise de devoir, à l'endroit indiqué par votre professeur.
}

\textbf{Pour obtenir de l'aide, rendez-vous à la page \pageref{correction_scratch2}}


\subsection{Pour aller plus loin...}

S'il vous reste du temps, essayez de dessiner une spirale à angles droits, puis une véritable spirale.

\uneimageici{./images/scratch/ScratchSpirales}{.5\textwidth}

\prof{Pour dessiner la spirale, il faut utiliser une variable qui permet d'augmenter le nombre de pas avant chaque changement de direction. Le but n'est pas forcément que les élèves parviennent à dessiner la spirale. Elle est en effet le sujet de l'activité 1 du livret de 5e (introduction de la notion de variable).}


\newpage


%
%
%  S  É  A  N  C  E     III
%
%









\section{Séance 3 : créer un petit jeu en Scratch}\label{ficheScratch3}

\vspace{20pt}

\subsection{Pour préparer l'activité...}

Voici une vidéo pour apprendre à modifier le costume, l'arrière-plan et rajouter de la musique avec \texttt{Scratch}. Pensez à la regarder avant de commencer l'activité de la séance :

\begin{center}
\qrcode[hyperlink, height=0.9in]{https://web.microsoftstream.com/video/d66b5112-a7c7-4be7-9a6c-7fcca1326531}
\end{center}

\vspace{20pt}

\subsection{Pour bien démarrer...}

Dès que vous avez ouvert un nouveau document dans \emph{Scratch}, sauvegardez-le au format \texttt{Nom-seance3.sb3} (dans le menu \texttt{Fichier}, choisir \texttt{Sauvegarder sur votre ordinateur}). Pendant que vous travaillez, pensez à sauvegarder régulièrement votre travail.   

%\uneimageici{./images/generales/clavierCmdS}{.4\textwidth}

\vspace{20pt}

\subsection{Sujet de l'activité...}

\vspace{10pt}
\prof{cette séance 3 est très longue. Il est probable que la majorité des élèves n'en atteignent pas la fin. Ce n'est pas gênant : 5 minutes avant la fin de la séance, demandez aux élèves d'enregistrer leur travail et de le rendre, quelque soit son avancement.}

\boiteEnonce{L'objectif de cette activité est de créer un petit jeu en \emph{Scratch}. Le but est simple : piloter un avion tout en évitant des obstacles. Voici à quoi ressemble ce jeu :}
\newpage
\boiteEnonce{\uneimageici{./images/scratch/jeuAvion}{.4\textwidth}
Pour programmer ce jeu, vous devrez procéder par étapes :
\begin{enumerate}
\item ajouter un nouvel objet : l'avion ;
\item faire avancer l'avion dans la scène de manière continue, et rebondir lorsqu'il touche les bords ;
\item ajouter la gestion des touches du clavier pour piloter l'avion ;
\item ajouter un nouvel objet : l'obstacle ;
\item ajouter la gestion de la collision entre l'avion et l'obstacle.
\end{enumerate}\vspace{0.5cm}
Attention, ce programme n'est pas si facile. Il va falloir réfléchir un peu avant de construire le code.\\\\
Une fois votre programme terminé, vous devrez l'enregistrer au format SB3 (le fichier doit être nommé à partir de votre nom : \texttt{Nom-seance3.sb3}) et le rendre sur Teams dans le dossier de remise de devoir, à l'endroit indiqué par votre professeur.}

\textbf{Pour obtenir de l'aide, rendez-vous à la page \pageref{correction_scratch3}}

\newpage

\subsection{Pour aller plus loin...}  

Si vous avez du temps, améliorez votre jeu. Vous pouvez par exemple :

\begin{itemize}
\item ajouter une action aux flèches \emph{haut} et \emph{bas} (par exemple respectivement avancer de 5 pas et avancer de $-5$ pas) ;  
\item ajouter d'autres obstacles à éviter (voir image à gauche ci-dessous) ;
\item modifier la scène pour qu'elle représente un ciel (voir image à droite ci-dessous).
\end{itemize}

\deuximagesici{./images/scratch/jeuAvionV2}{.6\textwidth}%
              {./images/scratch/jeuAvionV3}{.6\textwidth}
