\chapter*{Philosophie du document}


\prof{\textbf{ceci est la version professeur du document.} L'icône du professeur est suivie par des informations complémentaires qui n'apparaissent pas dans la version élève.}  

Vous avez entre les mains le deuxième tome d'une série de trois fascicules qui accompagneront les élèves des classes de 6\up{e}, 5\up{e}, 4\up{e} ET 3\up{e} jusqu'au moment où ils recevront un ordinateur qu'ils seront en mesure d'exploiter au mieux pour leur travail.

\vspace{18pt}

Ce document se présente sous la forme d'un livret qui rassemble des fiches MITIC\footnote{MITIC : Médias, Images et Technologies de l'Information et de la Communication.} permettant aux élèves d'apprendre à utiliser les logiciels et espaces numériques mis à leur disposition. Pour l'année de 5\up{e}, sont traités les logiciels \emph{Microsoft Word} (traitement de texte), \emph{Microsoft Excel} (tableur grapheur), \emph{Gimp} (retouche d'image), \emph{Audacity} (traitement des fichiers son) et \emph{Scratch} (programmation). Au début de chaque chapitre un lien permettant de télécharger le logiciel est fourni.

\vspace{18pt}


Chaque fiche est conçue pour être exploitée à plusieurs occasions et dans des matières différentes, à chaque fois lors d'une séance de 45 minutes. La fiche sur le tableur, par exemple, est découverte en physique-chimie (\emph{Séance 1}), exploitée à nouveau en mathématiques (\emph{Séance 2}) puis en histoire-géographie (\emph{Séance 3}) selon un calendrier proposé en début de fiche. Nous avons à chaque fois essayé de faire coïncider les notions abordées dans la fiche avec le programme de la matière concernée.

\vspace{18pt}

Professeurs, c'est à vous que revient la tâche délicate d'inclure le contenu de ces fiches dans votre progression. À vous de le faire vivre : arriver en salle informatique et demander aux élèves de remettre en forme un texte de Molière ne présente que peu d'intérêt pédagogique. Donnez du sens à ces fiches et profitez-en pour diversifier votre enseignement. N'hésitez pas à exploiter dans vos cours les techniques présentées dans ce fascicule afin que les élèves utilisent plusieurs fois leurs nouvelles compétences et, par là-même, les pérennisent.

%\vspace{18pt}

%Ces fiches MITIC sont appelées à évoluer. N'hésitez pas à nous transmettre vos suggestions et nous signaler toute erreur relevée par courriel à l'adresse \texttt{flo-mitic@florimont.ch}.

\vspace{18pt}

Merci d'avance à tous pour votre implication.

\vspace{18pt}

L'équipe de rédaction.

\vspace{2cm}

% \emph{Remarque : il existe une version professeur de ce document, contenant des informations complémentaires, disponible sur l'ENT de l'école.}

  
